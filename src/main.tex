\documentclass[12pt,openright,a4paper,brazil]{abntex2}

% ---
% Pacotes fundamentais 
% ---
\usepackage{lmodern}			% Usa a fonte Latin Modern
\usepackage[T1]{fontenc}		% Selecao de codigos de fonte.
\usepackage[utf8]{inputenc}		% Codificacao do documento (conversão automática dos acentos)
\usepackage{indentfirst}		% Indenta o primeiro parágrafo de cada seção.
\usepackage{color}				% Controle das cores
\usepackage{graphicx}			% Inclusão de gráficos
\usepackage{microtype} 			% para melhorias de justificação
% ---

% ---
% Pacotes adicionais, usados apenas no âmbito do Modelo Canônico do abnteX2
% ---
\usepackage{lipsum}				% para geração de dummy text
% ---

% ---
% Pacotes de citações
% ---
\usepackage[brazilian,hyperpageref]{backref}	 % Paginas com as citações na bibl
\usepackage[alf]{abntex2cite}	% Citações padrão ABNT


% O tamanho do parágrafo é dado por:
\setlength{\parindent}{1.3cm}

% Controle do espaçamento entre um parágrafo e outro:
\setlength{\parskip}{0.2cm}  % tente também \onelineskip


\titulo{Uma Introdução a Haskell}
\autor{Gustavo Lopes Rodrigues, Lucas Santiago Oliveira, Pedro Souza, Thiago Henriques }
\local{Belo Horizonte}
\data{2020}
\instituicao{%
  Pontifícia Universidade Católica Minas Gerais
  }
\tipotrabalho{Trabalho de LIP}

\begin{document} 

    \selectlanguage{brazil}

    \frenchspacing

    \imprimircapa

    \tableofcontents*

    \textual

    \chapter{Introdução}
    \chapter{Histórico sobre a linguagem, com sua cronologia}
    \chapter{Paradigma a que pertence}
    \chapter{Características mais marcantes da linguagem}
    \chapter{Linguagem similares ou confrontantes}
    \chapter{Exemplo(s) de programa(s)}
    \chapter{Estudo de Caso, sobre o(s) tema(s)}
    \chapter{Considerações finais}

    \begin{apendicesenv}

        \partapendices
        \chapter{Teste}
    
    \end{apendicesenv}

\end{document}


