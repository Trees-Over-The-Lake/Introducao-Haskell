\documentclass[
  % -- opções da classe memoir --
  12pt,				         % tamanho da fonte
  oneside,			       % para impressão apenas no recto. Oposto a twoside
  a4paper,			       % tamanho do papel. 
  article,
  % -- opções da classe abntex2 --
  %chapter=TITLE,		   % títulos de capítulos convertidos em letras maiúsculas
  %section=TITLE,		   % títulos de seções convertidos em letras maiúsculas
  %subsection=TITLE,	 % títulos de subseções convertidos em letras maiúsculas
  %subsubsection=TITLE % títulos de subsubseções convertidos em letras maiúsculas
  % -- opções do pacote babel --
  english,		       	 % idioma adicional para hifenização
  brazil,			      	 % o último idioma é o principal do documento
]{abntex2}
\usepackage{meupacote}

\titulo{Uma Introdução a Haskell}
\autor{Gustavo Lopes Rodrigues \and Lucas Santiago \and Pedro Souza \and Thiago Henriques}
\local{Belo Horizonte}
\data{2020}
\instituicao{%
  Pontifícia Universidade Católica Minas Gerais
  }
\tipotrabalho{Trabalho de LIP}
% ---
% 
% ---
% informações do PDF
\makeatletter
\hypersetup{
     	%pagebackref=true,
		pdftitle={\@title}, 
		pdfauthor={\@author},
    	pdfsubject={Trabalho de Linguagem de programação em LaTeX},
	    pdfcreator={GLR, LSO, PS, THN},
		pdfkeywords={abnt}{latex}{abntex}{abntex2}{Linguagem de programação}, 
		colorlinks=true,       		% false: boxed links; true: colored links
    	linkcolor=black,          	% color of internal links
    	citecolor=blue,        		% color of links to bibliography
    	filecolor=magenta,      		% color of file links
		urlcolor=blue,
		bookmarksdepth=4
}
\makeatother

\makeindex

% ---
% Iniciando efetivamente o documento
% ---
\begin{document} 

    % Fazer com que as secções sejão subcapitulos
    \renewcommand{\thesection}{\noindent\arabic{chapter}.\arabic{section}}
    % ---
    % Selecionando linguagem
    % ---
    \selectlanguage{brazil}
    % ---
    % Retira espaço extra obsoleto entre as frases.
    % ---
    \frenchspacing
    % ---
    % Imprimir a capa 
    % ---
    \imprimircapa
    % ---
    % Imprimir a tabela de conteúdos(Sumário)
    % ---
    \pdfbookmark[0]{\contentsname}{toc}
    \tableofcontents*
    \cleardoublepage
    % ---
    % PARTE TEXTUAL
    % ---
    \textual
    % ---
    % Criar nova página e então iniciar a escrita
    % ---
    \newpage
    
    \import{./seccoes}{chapter1.tex}

    \import{./seccoes}{chapter2.tex}

    \import{./seccoes}{chapter3.tex}

    \import{./seccoes}{chapter4.tex}

    \import{./seccoes}{chapter5.tex}

    \import{./seccoes}{chapter6.tex}

    \import{./seccoes}{conclusao.tex}

    \postextual

    \bibliography{referencias}
    \import{./seccoes}{appendice.tex}

    \phantompart

    \printindex

\end{document}


