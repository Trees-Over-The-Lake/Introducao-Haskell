\chapter{Linguagem similares ou confrontantes}

\begin{itemize}
  \item Prolog
  
  * Haskell eh uma linguagem funcional, que requer uma descrição matemática e lógica para resolver um problema. Prolog é uma linguagem declarativa, 
  que declara um conjunto de regras sobre qual a saída que deve ser resultada de qual entrada e assim deduz qual a saída deve ser o resultado de qual entrada;

  * Prolog é especialmente fácil para a resolução de problemas lógicos e Haskell é melhor para a resolução de problemas que podem ser computados com algoritmos.

  \item LISP 
  
  * Ambos Haskell e LISP têm uma sintaxe minimalista comparada a C++, C\# e Java;

  * Em Haskell são escritos pequenos pedaços de código que são combinados pela linguagem;

  * LISP é uma linguagem homiconica, enquanto Haskell usa computação monadica.

  \item Scheme 
  
  * Scheme, assim como Haskell, também é uma linguagem funcional;

  * A sintaxe de Scheme é extremamente simples e a linguagem se apoia em ter um núcleo com poucas funcionalidades que pode ser 
  facilmente expandida com diversas ferramentas de extensão;

  * Scheme é uma linguagem interpretada, que se difere de Haskell, já que Haskell tem a opção de ser tanto uma linguagem compilada ou interpretada;

  * Scheme é uma linguagem que utiliza tipagem dinâmica, o que entra em constraste com Haskell, que utiliza tipagem estática.

  \item ML 
  
  * ML é uma linguagem que usa declaração implícita, o que garante \emph{type-safety}. Haskell não possui declaração implícita;

  * Ambas possuem coletor de lixo;

  * ML é uma linguagem que utiliza chamada por valor, enquanto Haskell possui tanto chamada por valor quanto por referência.

  \item Miranda 
  
  * Ambas são linguagens puramente funcional. Linguagens puramente funcionais executam todo o código em funções;

  * Miranda utiliza tipagem estática, assim como Haskell, C e outras linguagem da época;

  * Miranda tem um algoritmo de parsing implementado que faz uso de indentação ao invés de chaves - { } - para indentação. 
  Essa característica viria a ser popularizada na linguagem Python muitos anos no futuro. Haskell, assim como muitas outras linguagens da época, 
  utiliza caracteres para indentação;

  * Por ser uma linguagem puramente funcional, é impossível existir \emph{side-effects} em ambas as línguas.
  
  \item Elixir 

* Elixir é compilada para bytecode que será rodado na Erlang Virtual Machine. Haskell é compilada para código de máquina;

* Elixir consegue chamar funções de Erlang e vice-versa, isso sem qualquer impacto em \emph{runtime}. Haskell não é capaz de chamar funções de outras linguagens, pois ela é compilada;

* Elixir é uma linguagem \emph{concurrent}, o que significa que o código é rodado ao mesmo tempo ao invés de sequencialmente. Haskell, pelo contrário, é uma linguagem sequencial, podendo ser paralelizada manualmente;

* Assim como Haskell, Elixir é uma linguagem funcional.


\end{itemize}

\newpage