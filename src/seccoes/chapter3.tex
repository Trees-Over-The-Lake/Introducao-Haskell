\chapter{Paradigma a que pertence}

    O paradigma oferece e determina a visão que o programador possui sobre a estruturação
    e a execução do programa. Um exemplo bem famoso de paradigma é o POO (Programação orientada a objetos),
    conhecido por ser o modelo para linguagens como C++ e Java.

    A Programação funcional é um paradigma que descreve uma expressão matemática a ser avaliada,
    mapeando dos valores de entradas nos valores de retorno, por meio de funções. Em outras palavras: 
    a programação funcional é baseado em cima de funções que rodam no topo de outras funções. Programação 
    funcional está englobada junto ao grupo da Programação descritiva.

    Eis mais algumas características deste paradigma:

    \section{Dados imutáveis} 

    É possível declarar valores a variáveis, mas não pode mudar o valor 
    delas durante a compilação. Isso acontece porque, diferente de linguagens imperativas(Como C), onde você
    atribui um valor e pode mudá-lo em execuções, variáveis em Haskell(por exemplo) possuem tipagem forte, logo 
    não sofrem efeitos colaterais(\emph{side effects}).

    \section{Funções puras}

    Também presentes em linguagens como JavaScript e Python, Funções puras são aquelas que recebem 
    um parâmetro \emph{input} e sempre vão retornar o mesmo \emph{input} sem causar efeitos colatreis
    ao programa.

    Esse tipo de função é positiva pelos seguintes aspectos:
    
    \begin{itemize}
      \item facilita execução de códigos em paralelo, pois não impactam outras funcionalidades que estão em atuação;
      \item Maior facilidade em criar cache, já que os mesmos parâmetros são sempre esperados.
    \end{itemize} 

    \newpage

    \section{Cálculo Lambda}
    
    Como mencionado anteriormente, o Cálculo Lambda está presente na programação funcional, e em Haskell não é diferente.
    Nessa lingua, é possível utilizar as chamadas \emph{Expressões lambda} que são funções anônimas(sem nome), formadas por 
    uma sequência de padrões:

    \begin{itemize}
      \item Argumentos da função
      \item Corpo
    \end{itemize}

    \begin{gather*}
      \text{Função anônima para calcular o dobro de x} \\ x \rightarrow x + x 
    \end{gather*}

    \section{Análise crítica} 

    As linguas que seguem o paradigma funcional se caracterizam em criar 
    funcionalidade dentro de estruturas de fácil compreensão e definição. Isso ajuda a manter 
    a confiabilidade, e leitura do código, principalmente quando integrado à forte tipagem. Além disso,
    táis códigos se caracterizam pelo alto nível de abstração, principalmente ao utilizar as funções,
    já que isso permite supressão de detalhes da programação, e garante uma menor probabilidade 
    de erros.

    Em contraste, tais ideias podem se tornar complicadas, quando utilizadas dentro de contextos
    onde é necessário de muitas variáveis(Exemplo: Banco de Dados). Além disso, por conta que as funções 
    possuem a característica de serem recursivas por padrão, elas tendem a ser mais demoradas para computar, 
    quando comparado as linguagens imperativas. Ainda nesse ponto, funções recursivas podem gerar maior uso de memória.

    %% Referencias dessa parte

    \nocite{haskellslides}
    \nocite{funcoespura}
    \nocite{funcoespura2}
    \nocite{haskellwikipedia}
    \nocite{abntex2-wiki-como-customizar}

    %%

    \newpage