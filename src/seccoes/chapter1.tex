\chapter*[Introdução]{Introdução}
    \addcontentsline{toc}{chapter}{Introdução}

    Em 1930, Alonzo Church, matemático estadunidense apresentou o Cálculo Lambda, como parte da investigação dos fundamentos da matemática. O Cálculo Lambda é um sistema que
    estuda funções recursivas computáveis e foi utilizada como base para as teorias e fundamentos matemáticos por trás do paradigma da Programação Funcional. Ele também
    pode ser considerado a primeira linguagem programação funcional, todavia, não foi projetada para ser executada em computadores, sendo apenas um modelo que descreve relações entre funções
    simples, permitindo criar funções mais complexas.

    Com o passar dos anos, varias linguagens funcionais foram criadas, sendo alguns exemplos a linguagem LISP em 1955 e a ML no final da década de 70. Porém, não
    havia um padrão para as linguagens desse paradigma, e quando chegou a segunda metade da década de 80, havia uma necessidade de criar uma única linguagem, que englobasse
    as melhores práticas de projeto, além de implementar as técnicas funcionais que estavam em alta na época.

    Nesse contexto Haskell foi criado em 1987, por Peyton Jones e Paul Hudak. Sendo assim a The Yale Meeting foi a primeira reunião presencial, no qual foi decidido
    os principais objetivos que a linguagem proporcionaria, como também a escolha do nome. 
    Segue as metas estabelecidas na reunião:

    \begin{itemize}
      \item Ser viável para o ensino, pesquisa e aplicações, incluindo sistema de larga escala;
      \item Ser completamente descritiva via publicação no tocante à sua sintaxe e sua semântica;
      \item Não ser proprietária, tal que qualquer um pudesse implementá-la e distribuí-la;
      \item Basear-se em ideias que envolvessem o senso comum;
      \item Reduzir a diversidade desnecessária de outras linguagens funcionais.
    \end{itemize} 

    A implementação de Haskell começou do zero, desenvolvendo funções únicas e tendo inspiração na linguagem Miranda que estava desempenhando um papel 
    importante na época. Em geral, essa linguagem passou por algumas versões que ajudou muito no desempenho e na adição de novas funções. 

    \newpage