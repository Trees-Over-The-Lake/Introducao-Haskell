\documentclass[aspectratio=169]{beamer}	 	

\usetheme{Goettingen}
\usecolortheme{default}
\usefonttheme{professionalfonts}			% para fontes matemáticas
% Enconte mais temas e cores em http://www.hartwork.org/beamer-theme-matrix/ 
% Veja também http://deic.uab.es/~iblanes/beamer_gallery/index.html

% Customizações de Cores: fg significa cor do texto e bg é cor do fundo

\definecolor{mypurple}{RGB}{75, 0, 130} % changed this
\definecolor{mypurple2}{RGB}{44, 0, 73} % changed this
\definecolor{mygray}{RGB}{206, 206, 206} % changed this
\definecolor{myblue}{RGB}{0, 0, 139} % changed this

\setbeamercolor{normal text}{fg=black}
\setbeamercolor{alerted text}{fg=red}
\setbeamercolor{author}{fg=blue}
\setbeamercolor{institute}{fg=blue}
\setbeamercolor{date}{fg=myblue}
\setbeamercolor{frametitle}{fg=mypurple}
\setbeamercolor{framesubtitle}{fg=black}
\setbeamercolor{block title}{bg=mypurple2, fg=white}		%Cor do título
\setbeamercolor{block body}{bg=mygray, fg=darkgray}	%Cor do texto (bg= fundo; fg=texto)
\setbeamercolor{structure}{fg=mypurple}

% ---
% PACOTES
% ---
\usepackage[alf]{abntex2cite}	% Citações padrão ABNT
\usepackage[brazil]{babel}		% Idioma do documento
\usepackage{color}			      % Controle das cores
\usepackage[T1]{fontenc}		  % Selecao de codigos de fonte.
\usepackage{graphicx}			    % Inclusão de gráficos
\usepackage[utf8]{inputenc}		% Codificacao do documento (conversão automática dos acentos)
\usepackage{txfonts}			    % Fontes virtuais
% ---

% --- Informações do documento ---
\title{Apresentando Haskell}
\author{Gustavo Lopes \and Lucas Santiago \and Pedro Souza \and Thiago Henriques }
\institute{Pontifícia Universidade Católica de Minas Gerais}
\date{26 de Março de 2021}
% ---

% ----------------- INÍCIO DO DOCUMENTO --------------------------------------
\begin{document}

    % ----------------- NOVO SLIDE --------------------------------
    \begin{frame}

    \begin{minipage}{1\linewidth}
      \centering
      \begin{tabular}{cc}
        \begin{tabular}{c}
          \includegraphics[width=3.0cm]{Haskell.png}
        \end{tabular}
      \end{tabular}
    \end{minipage}

    \titlepage

    \end{frame}

    % ----------------- NOVO SLIDE --------------------------------
    %TABELA DE CONTEÚDO

    \AtBeginSection[]
    {
    \begin{frame}
      \frametitle{Conteúdo}
    \tableofcontents[currentsection]
    \end{frame}
    }

    % ----------------- NOVO SLIDE --------------------------------
    \section{Introdução}

    \begin{frame}{Introdução}

    

    \end{frame}

    % ----------------- NOVO SLIDE --------------------------------
    \section{Histórico sobre a linguagem}
    \begin{frame}

      \frametitle{Histórico sobre a linguagem}
      \framesubtitle{com sua cronologia}

    \end{frame}

    % ----------------- NOVO SLIDE --------------------------------
    \section{Paradigma}

    \begin{frame}
      \frametitle{ABNT}
      \framesubtitle{Normas para trabalhos acadêmicos}
    \end{frame}

    % ----------------- NOVO SLIDE --------------------------------
    \section{Características mais marcantes}

    \begin{frame}
      \frametitle{abnTeX2}
      \framesubtitle{Usando a suíte abnTeX2}
    \end{frame}

    % ----------------- NOVO SLIDE --------------------------------
    \section{Linguagem similares}

    \begin{frame}
      \frametitle{abnTeX2}
      \framesubtitle{Usando a suíte abnTeX2}
    \end{frame}

    % ----------------- NOVO SLIDE --------------------------------
    \section{Exemplo(s) de programa(s)}

    \begin{frame}
      \frametitle{abnTeX2}
      \framesubtitle{Usando a suíte abnTeX2}
    \end{frame}

    % ----------------- NOVO SLIDE --------------------------------
    \section{Considerações finais}

    \begin{frame}
      \frametitle{abnTeX2}
      \framesubtitle{Usando a suíte abnTeX2}
    \end{frame}

    % ----------------- NOVO SLIDE --------------------------------
    \section{Referências}

    % --- O comando \allowframebreaks ---
    % Se o conteúdo não se encaixa em um quadro, a opção allowframebreaks instrui 
    % beamer para quebrá-lo automaticamente entre dois ou mais quadros,
    % mantendo o frametitle do primeiro quadro (dado como argumento) e acrescentando 
    % um número romano ou algo parecido na continuação.

    \begin{frame}[allowframebreaks]{Referências}
      \bibliography{abntex2-modelo-references}
    \end{frame}

% ----------------- FIM DO DOCUMENTO -----------------------------------------
\end{document}